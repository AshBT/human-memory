\documentclass[11pt]{article}
\usepackage{amsmath}
\usepackage{setspace}
\usepackage{pxfonts}
%\usepackage{graphicx}
\usepackage{geometry}


\geometry{letterpaper,left=.5in,right=.5in,top=0.5in,bottom=.75in,headsep=5pt,footskip=20pt}

\title{PSYC 51.09: Problem Set 3}
%\author{Jeremy R. Manning}
\date{}

\begin{document}
\maketitle
\vspace{-0.75in}
\section*{Introduction}
This problem set is intended to solidify the concepts you learned
about in this week’s lectures and readings.  Your responses will be
worth 5\% of your final grade.  \textit{After attempting each problem
  on your own,} you are encouraged to work together with your classmates in small groups, and/or to post and answer questions on the course’s Canvas site.  \textbf{\textit{However, you must clearly indicate who your collaborated with and submit your own (uniquely worded) responses.}}

We will go over the answers to this problem set in class on
\textbf{Tuesday, October 15, 2019 at 10:10 am}.  You must upload your answers before then in order to receive credit.  No late submissions will be accepted.

\section*{Readings}
\begin{enumerate}
\item Read Chapter 3 of \textit{Foundations of Human Memory}.  What were your thoughts on the reading?  For example, did you learn something interesting?  Were you surprised by something?  Do you disagree with the author?  Did you think some concept was described especially well (or confusingly)?  \textbf{(Ungraded)}
\item Blei (2012) describes a ``topic model,'' which is a computational algorithm that can mathematically discover and describe “topics” or “categories” by analyzing a large collection of documents.  Read the paper (skimming or skipping the equations and the descriptions of “graphical models”).  What did you think?  \textbf{(Ungraded)}
\item Mitchell et al (2008), Huth et al. (2012), and Huth et
  al. (2016) all describe techniques for decoding which word someone
  is thinking of using their brain activity (recorded using a
  technique called functional magnetic resonance imaging, or fMRI).
  This technique allows researchers to obtain a 3D “snapshot” of
  someone’s brain activity about once per second during an experiment.
  Read one or more of these papers to get the high-level ideas (skimming the equations or any
  other methods that you don't understand).  What did you think?  What
  are the ethical implications of these papers?  You may also be
  interested in checking out this web demonstration of where in the
  brain different concepts are represented:
  https://gallantlab.org/huth2016/.  \textbf{(Ungraded)}
\end{enumerate}

\section*{Graded questions}
Show all work and provide answers rounded to the nearest thousandth (third
decimal place).
\begin{enumerate}
\item After studying a list of 4 items during a trial of a Sternberg
  task, a participant's memory matrix $M$ is given by

\begin{eqnarray*}
M&=&
\left( 
\begin{array}{cccc}
2 & 2 & 5 & 1\\
4 & 5 & 6 & 4\\
6 & 1 & 2 & 7\\
8 & 1 & 6 & 9\\
10 & 2 & 5 & 8
\end{array} 
\right) \\
&=&
\left( 
\begin{array}{cccc}
\mathbf{m}_1 & \mathbf{m}_2 & \mathbf{m}_3 & \mathbf{m}_4  
 \end{array} 
\right) \nonumber
\end{eqnarray*}
\begin{enumerate}
\item \textbf{(1.5 pt)} Suppose the probe item is the target $\mathbf{m}_3$.  Compute
  the similarities to $\mathbf{m}_1$, $\mathbf{m}_2$, and
  $\mathbf{m}_4$ using \textbf{distance-based similarity} (let $\tau =
  0.25$):
\[
\mathrm{similarity}(\mathbf{a}, \mathbf{b}) = e^{-\tau||\mathbf{a} -
  \mathbf{b}||} = e^{-\tau\sqrt{\sum_{i=1}^N(a(i) - b(i))^2}}.
\]
\item \textbf{(1.5 pt)}  Compute the same similarity values using
  \textbf{cosine-based similarity}:
\begin{align}
\cos\theta(\mathbf{a}, \mathbf{b}) &=
\frac{\mathbf{a}\cdot\mathbf{b}}{||\mathbf{a}||~||\mathbf{b}||},~\mathrm{where}\notag\\
\mathbf{x}\cdot\mathbf{y} &= \sum_{i=1}^N x(i)y(i),~\mathrm{and}\notag\\
||\mathbf{c}|| &= \sqrt{\sum_{i=1}^N c(i)^2}\notag\\
\end{align}

\item \textbf{(1 pt)} Consider your calculations in (a) and (b).  Would
  choosing one similarity metric (versus the other) have led to any
  different predictions about similarity or memory judgements?
  Explain (1-2 paragraphs).

\item \textbf{(1 pt)}  Suppose the participant instead receives the probe item
  $\mathbf{p}_1$:
\begin{eqnarray*}
\mathbf{p}_1&=&
\left( 
\begin{array}{c}
1\\
5\\
6\\
8\\
9
\end{array} 
\right).
\end{eqnarray*}
If the participant's summed similarity decision threshold is 1.5, do you predict
that they'll call the probe item old or new?  Use distance-based
similarity (with $\tau = 0.25$) to justify your prediction.


\end{enumerate}


\end{enumerate}

\end{document}


