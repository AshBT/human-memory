\documentclass[11pt]{article}
\usepackage{amsmath}
\usepackage{setspace}
\usepackage{pxfonts}
%\usepackage{graphicx}
\usepackage{geometry}


\geometry{letterpaper,left=.5in,right=.5in,top=0.5in,bottom=.75in,headsep=5pt,footskip=20pt}

\title{PSYC 51.09: Problem Set 7}
%\author{Jeremy R. Manning}
\date{}

\begin{document}
\maketitle
\vspace{-0.75in}
\section*{Introduction}
This problem set is intended to solidify the concepts you learned about in this week's lectures and readings.  Your responses will be worth 3\% of your final grade.  You are encouraged to work together with your classmates in small groups, and/or to post and answer questions on the course’s Canvas site.  \textbf{\textit{However, you must clearly indicate who your collaborated with and submit your own (uniquely worded) responses.}}

We will go over the answers to this problem set in class on \textbf{Monday, February 29, 2016 at 1:45 pm}.  You must upload your answers before then in order to receive credit.  No late submissions will be accepted.

\section*{Readings and ungraded questions}
\begin{enumerate}
\item Read Chapter 8 of \textit{Foundations of Human Memory}.  What were your thoughts on the reading?
  \textbf{(Ungraded)}

\item Read Chapter 9 of \textit{Foundations of Human Memory}.  What were your thoughts on the reading?
  \textbf{(Ungraded)}

\item Create an outline of your final paper.  I'll provide feedback
  within a few days.  You can either include it as part of your
  problem set submission,
  or send it to me separately via email.
  \textbf{(Ungraded, Optional)}
\end{enumerate}

\section*{Graded questions}
In answering the questions below, consider this week's material in the
context of the other material we've learned throughout the course.

\begin{enumerate}
\item How do our brains organize and spontaneously retrieve memories?
  Use an example if it helps, or you can give a general answer.
  \textbf{(2-3 paragraphs, 1 point)}

\item What do you see as the single greatest challenge to our understanding
  of human memory?  For example, where is our knowledge of ``how
  memory works'' weakest?  Or, what sorts of questions about memory are
  the most difficult to study?  Why?  \textbf{(2-3 paragraphs, 1 point)}

\item What would need to happen in order to overcome (solve) the challenge you
  identified above?  Do you think it's possible and/or will ever be
  possible to address that challenge?  Why?  \textbf{(2-3 paragraphs, 1 point)}
\end{enumerate}

\end{document}


